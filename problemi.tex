\section{Problemi Basic}
\subsection{GB - 1, Problemi}
\subsection{GB - 2, Problemi}
\subsection{GB - 3, Problemi}
\begin{enumerate}
    \item \textbf{Polish MO 2018 - 5} Sia $ABC$ un triangolo acutangolo con $AB\neq AC$
    e siano $E,F$ i piedi delle altezze su $AC$ e $AB$. La tangente in $A$ alla circoscritta interseca $BC$ in $P$. La retta parallela a $BC$ passante per $A$ interseca $EF$ in $Q$. 
    
    Dimostrare che $PQ$ è perpendicolare alla mediana passante per $A$ del triangolo $ABC$
    
    %Assi radicali swag: 1) La circonferenza degenere di centro $A$, la circoscritta a $AEF$ e a $BCEF$ hanno $Q$ come centro radicale (in quanto sta su $EF$ per le ultime due e $AQ$ tange la circoscritta $AEF$ per le prime due). 2) $PA^2=PB\cdot PC$, quindi P sta sull'asse radicale tra $A$ e la circoscritta a $BCEF$. Dunque $PQ$ è asse radicale delle due circonferenze ed è perpendicolare alla congiungente dei centri, che è AM
\end{enumerate}


\clearpage

\section{Problemi Medium}
\subsection{GM - 1, Problemi}
\begin{enumerate}
	\item \textbf{[BMO 2009 - 2]} Sia $MN$ una segmento parallelo al lato $BC$ del triangolo $ABC$, con $M$ sul lato $AB$ e $N$ sul lato $AC$. Le rette $BN$ e $CM$ si incontrano in $P$. Le circonferenze circoscritte a $BMP$ e $CNP$ si incontrano in due punti distinti $P$ e $Q$. 
	
	Mostrare che $\angle BAQ = \angle CAP$.
	\item \textbf{[RMM 2012 - 2]} Sia $ABC$ un triangolo non isoscele e siano $D$, $E$ e $F$ rispettivamente i punti medi dei lati $BC$, $CA$ e $AB$. La circonferenza $BCF$ e la retta $BE$ si intersecano nuovamente in $P$ e la circonferenza $ABE$ e la retta $AD$ in $Q$. Le rette $DP$ e $FQ$ si incontrano in $R$. 
	
	Mostrare che il baricentro $G$ del triangolo $ABC$ giace sulla circonferenza circoscritta al triangolo $PQR$.
	\item \textbf{[USAMO 2016 - Day 2 - 2]} Un pentagono equilatero $AMNPQ$ è inscritto in un triangolo $ABC$ in modo che $M\in AB$, $Q\in AC$ e $N,p \in BC$. Sia $S$ l'intersezione di $MN$ e $PQ$ e denotiamo con $l$ la bisettrice di $\angle MSQ$. 
	
	Mostrare che, detto $I$ l'incentro di $ABC$, $OI$ è parallelo a $l$.
	\item \textbf{[IMO 2008 - 6]} Sia $ABCD$ un quadrilatero convesso con $BA \neq BC$. Siano $\omega_1$ e $\omega_2$ le circonferenze inscritte ai triangoli $ABC$ e $ADC$ rispettivamente. Supponiamo che esista una circonferenza $\omega$ tangente alla retta $BA$ oltre A, alla retta $BC$ oltre $C$, alla retta $AD$ e alla retta $CD$.
	
	Mostrare che le tangenti esterne comuni a $\omega_1$ e $\omega_2$ si intersecano su $\omega$.
	
	\item \textbf{[BMO 2015 - 2]} Sia $ABC$ un triangolo scaleno con incentro $I$ e circonferenza circoscritta $\omega$. $AI$, $BI$ e $CI$ intersecano $\omega$ di nuovo nei punti $D$, $E$ e $F$ rispettivamente. Le rette parallele a $BC$, $CA$ e $AB$ condotte da $I$ intersecano $EF$, $DF$ e $DE$ rispettivamente nei punti $K$, $L$ e $M$.
	
	Mostrare che $K$, $L$ e $M$ sono allineati.
	
	\item \textbf{[IMO 2012 - 1]} Dato un triangolo $ABC$, sia $J$ il centro della circonferenza ex-inscritta opposta al vertice $A$, la quale tange $BC$ in $M$ e le rette $AB$ e $AC$ in $K$ e $L$ rispettivamente. Le rette $LM$ e $BJ$ si intersecano in $F$ e le rette $KM$ e $CJ$ si intersecano in $G$. Sia $S$ il punto d'intersezione fra $AF$ e $BC$ e sia $T$ il punto d'intersezione fra $AG$ e $BC$. 
	
	Mostrare che $M$ è il punto medio di $ST$.
	\item \textbf{[IMO SL 2011 - 4]} Sia $ABC$ un triangolo acutangolo scaleno, e sia $\gamma$ la sua circonferenza circoscritta.
	Siano $A_0$ il punto medio di BC, $B_0$ il punto medio di $AC$ e $C_0$ il punto medio di $AB$. Sia
	$D$ il piede dell’altezza uscente da $A$, $D_0$ la proiezione di $A_0$ sulla retta $B_0C_0$ e $G$ il
	baricentro di $ABC$. Sia $\gamma_1$ la circonferenza passante per $B_0$ e $C_0$, e tangente a $\gamma$ in un
	punto $P$ diverso da $A$.
	\begin{itemize}
	\item Dimostrare che la retta $B_0C_0$ e le tangenti a $\gamma$ nei punti $A$ e $P$ sono concorrenti.
	\item Dimostrare che i punti $D_0$, $G$, $D$, e $P$ sono allineati.
	\end{itemize}
	\item \textbf{[USA TST 2012 - December Test - 1]} In un triangolo acutangolo $ABC$ si ha $\angle A<\angle B$ e $\angle A<\angle C$. Sia $P$ un punto variabile su $BC$. I punti $D$ e $E$ giacciono su $AB$ e $AC$ rispettivamente in modo che $BP=PD$ e $CP=PE$.
	
	Mostrare che al variare di $P$ sul segmento $BC$, la circonferenza circoscritta al triangolo $ADE$ passa per un punto fisso oltre $A$.
\end{enumerate}
\clearpage
\subsection{GM - 2, Problemi}
\begin{enumerate}
	\item \textbf{[China NMO 2017 - 2]} Siano $\omega$ e $\Omega$ di centro $I$ e $O$ rispettivamente la circonferenza inscritta e circoscritta a un triangolo acutangolo
	$ABC$. La circonferenza $\omega$ interseca $BC$ in $D$ e le tangenti a $\Omega$ passanti per $B$ e $C$ si intersecano in $L$.
	Siano $AH$ l'altezza condotta da $A$ a $BC$ e $X$ l'intersezione di $AO$ con $BC$. Siano $P$ e $Q$ le 
	intersezioni di $OI$ con $\Omega$.
	
	Mostrare che $PQXH$ è ciclico se e solo se $A,D$ e $L$ sono allineati.
	\item \textbf{[IMO 2014 - 4]} Siano $P$ e $Q$ punti su un segmento $BC$ di un triangolo acutangolo $ABC$ tali che $\angle PAB = \angle BCA$ e $\angle CAQ=\angle ABC$. Siano $M$ e $N$ punti su $AP$ e $AQ$ rispettivamente tali che $P$ è punto medio di $AM$ e $Q$ è punto medio di $AN$.
	
	Mostrare che l'intersezione di $BM$ e $CN$ giace sulla circonferenza circoscritta di $ABC$.
	\item \textbf{[Iran TST 2007 - Day 2 - 3]}
	Sia $\omega$ la circonferenza inscritta ad un triangolo $ABC$ che tange $AB$ e $AC$ rispettivamente in $F$ e $E$. Siano $P$ e $Q$ su $AB$ e $AC$ rispettivamente in modo che $PQ$ sia parallelo a $BC$ e tangente ad $\omega$. Siano $T$ l'intersezione di $EF$ con $BC$ e $M$ il punto medio di $PQ$. 
	
	Mostrare che $TM$ tange $\omega$.
	
	%Se X=AD\cap \omega, $TX$ tange $\omega$ per quadrilateri armonici. Poi (XDAY)=-1 e proiettando da $T$ su $PQ$ ottengo che l'intersezione di $TX$
	% con $PQ$ è il suo punto medio
	
	\item \textbf{[Iran TST 2009 - Day 2 - 3]}
	In un triangolo $ABC$ è inscritta una circonferenza $\omega$ di centro $I$ che interseca i lati $BC$, $CA$ e $AB$ rispettivamente in $D$, $E$ e $F$. Sia $M$ il piede della perpendicolare da $D$ a $EF$. Sia $P$ il punto medio di $DM$ e $H$ l'ortocentro del triangolo $BIC$.
	
	Mostrare che $PH$ biseca $EF$. 
	\item \textbf{[Romania TST 2007 - Day 7 - 2]}	La circonferenza inscritta al triangolo $ABC$ è tangente 
	ad $AB$ e $AC$ in $F$ ed $E$ rispettivamente. Sia $M$ il punto di $BC$ e $N$ l'intersezione di $AM$ con $EF$. La circonferenza di diametro $BC$ interseca $BI$ e $CI$ in $X$ e $Y$ rispettivamente.
	
	Mostrare che $\displaystyle\frac{NX}{NY}=\displaystyle\frac{AC}{AB}$.
	
	%Usa l'esercizio 13 e nota che DXY è simile ad ABC e ID è bisettrice di YDX. Oppure semplicemente formula seni-lati su IXY e un po' di trigonometria
	
	\item \textbf{[IMO SL 2007 - G8]}
	Sul lato $AB$ di un quadrilatero convesso $ABCD$ è preso un punto $P$. Sia $\omega$ la circonferenza inscritta al triangolo $CPD$ e sia $I$ il suo centro. Supponiamo che $\omega$ sia tangente alle circonferenze inscritte ai triangoli $APD$ e $BPC$ in $K$ e $L$ rispettivamente. Siano $E$ l'intersezione delle rette $AC$ e $BD$ e $F$ l'intersezione delle rette $AK$ e $BL$.
	
	Mostrare che $E$, $I$ e $F$ sono allineati.
	

\end{enumerate}
\clearpage
\subsection{GM - 3, Problemi}
\begin{enumerate}
	\item \textbf{[USA TST 2007 - 5]} Il triangolo $ABC$ è inscritto in una circonferenza $\Gamma$. Le tangenti a $\Gamma$ condotte da $B$ e $C$ si intersecano in $T$. Il punto $S$ è sulla retta $BC$ dimodoché $AS\perp AT$. Siano $B_1$ e $C_1$ sulla retta $ST$ dimodoché $B_1T=BT=C_1T$.
	
	Mostrare che $ABC$ e $AB_1C_1$ sono simili.
	\item \textbf{[IMO 2005 - 5]}
	Sia $ABCD$ un quadrialtero convesso con $BC=DA$ e $BC$ non parallelo a $DA$. Siano $E$ e $F$ su $BC$ e $DA$ rispettivamente tali che $BE=DF$. Siano $P$ l'intersezione di $AC$ e $BD$, $Q$ l'intersezione di $BD$ e $EF$ e $R$ l'intersezione di $EF$ e $AC$.

	Mostra che, al variare di $E$ e $F$, la circonferenza circoscritta al triangolo $PQR$ passa per un punto fisso (oltre $P$). 
	
	\item \textbf{[?]}  Sia $ABC$ un triangolo e siano $D$ e $E$ i piedi delle altezze relative ad $A$ e $B$, rispettivamente,  le quali siintersecano  in $H$.   Sia $M$ il  punto  medio  di $AB$ e  supponiamo  che  le  circonferenze circoscritte a $ABH$ e $DEM$ si intersechino nei punti $P$ e $Q$ (con $P$ e $A$ sullo stesso lato di $CH$).
	%PreIMO Mattino 4 2016
	
	Mostrare che le rette $PH$ e $MQ$ si incontrano sulla circonferenza circoscritta ad $ABC$. 
	\item \textbf{[IMO SL 2006 - 9]}
	Sui lati $BC$, $CA$ e $AB$ di un triangolo $ABC$ si scelgano tre punti $A_1$, $B_1$ e $C_1$ rispettivamente. Le circonferenze circoscritte a $AB_1C_1$, $BC_1A_1$ e $CA_1B_1$ intersecano la circonferenza circoscritta ad $ABC$ in $A_2$, $B_2$ e $C_3$ rispettivamente. Siano, inoltre, $A_3$, $B_3$ e $C_3$ rispettivamente i simmetrici di $A_1$, $B_1$ e $C_1$ rispetto ai punti medi dei lati del triangolo su cui giacciono. 
	
	Mostrare che i triangoli $A_2B_2C_2$ e $A_3B_3C_3$ sono simili.
	
	%A_2 è il centro della spilar similiarity che porta BC_1 in CB_1 quindi A_2C/A_2B=B_1C/C_1B=AB_3/AC_3 da cui A_2BC è
	%simile ad AC_3B_3 e da qui sono angoli 
	\item \textbf{[EGMO 2013 - 5]}
	Sia $\Omega$ la circonferenza circoscritta ad un triangolo $ABC$. La circonferenza $\omega$ è tangente ai lati $AC$ e $BC$ e internamente alla circonferenza $\Omega$ in un punto $P$. Una retta parallela ad $AB$ che interseca l'interno del triangolo $ABC$ è tangente a $\omega$ in $Q$.
	
	Mostrare che $\angle ACP = \angle QCB$.
	\item \textbf{[IMO SL 2003 - 4]}
	 Siano  $\Gamma_1$, $\Gamma_2$, $\Gamma_3$, $\Gamma_4$ 
	 circonferenze distinte tali che
	 $\Gamma_1$ e $\Gamma_3$ (così come $\Gamma_2$ e $\Gamma_4$) siano tangenti esternamente in $P$. Supponiamo che $\Gamma_1$ e $\Gamma_2$, $\Gamma_2$ e $\Gamma_3$, $\Gamma_3$ e $\Gamma_4$, $\Gamma_4$ e $\Gamma_1$ si intersechino in $A$, $B$, $C$ e $D$ rispettivamente e che nessuno di questi punti sia $P$.
	 
	 Mostrare che 
	 $$
	 \frac{AB\cdot BC}{AD\cdot DC}=\frac{PB^2}{PD^2} .
	 $$
	 \item \textbf{[IMO 2015 - 3]} Sia $ABC$ un triangolo acutangolo con $AB > AC$. Sia $\Gamma$ la sua circonferenza circoscritta, $H$ il suo ortocentro, e $F$ il piede dell'altezza condotta da $A$. Sia $M$ il punto medo di $BC$. Sia $Q$ il punto su $\Gamma$ tale che $\angle HQA = 90^{\circ}$ e sia $K$ il punto su $\Gamma$ tale che $\angle HKQ = 90^{\circ}$. Assumiamo che $A$, $B$, $C$, $K$ e $Q$ sono tutti distinti e giacciono su $\Gamma$ in quest'ordine. 
	 
	 Mostrare che le circonferenze circoscritte ai triangoli $KQH$ e $FKM$ sono fra loro tangenti.
	 %Inversione di centro H che fissa la circonferenza circoscritta ad ABC. K'Q' diviene perpendicolare ad AK' che è l'asse di F'M' e dunque K'Q' è la tangente a K' nella circonferenza circoscritta a F'M'K'.
		
\end{enumerate}
